\chapter*{}

\chapterprecishere{I have had it with all of these storytellers! They cannot write anything useful, pleasant, reassuring, all they do is uncover dirty secrets! I would ban them from writing! It is absolutely horrible: I get lost in what I am reading, and then all sorts of trash comes into my head; truly, I would ban them from writing; just ban them from writing anything at all.\par\raggedleft--- \textup{Prince V.F. Odoyevsky}}

\begin{flushright}
	April \nth{8}
\end{flushright}

\noindent My priceless Varvara Alekseyevna!

Yesterday I was happy, extremely happy, impossibly happy! For once in your life, you stubborn girl, you listened to me. It was eight o'clock in the evening, I am waking up (you know, precious, how I love to sleep for an hour or two after I leave the office), I got a candle, I am getting my papers in order, sharpening my quill, and suddenly, without any warning, I look up---honestly, my heart skipped a beat! You finally understood what I wanted, what my little heart wanted so badly! I see that the corner of your little curtain is turned up and hooked to your potted balsam plant, just exactly as I have been hinting you should do; just then I thought I could see your little face appearing in the window, too, that you were watching me from your little room, that you were thinking of me. How annoyed I was, my pet, that I could not see your little face very well! There was a time when even we saw clearly, precious. Old age is no walk in the park, my child! Now everything I see is hazy; I work a little in the evening, just copying something, and in the morning my eyes are all red and full of tears so that it is shameful to be seen by other people. But in my imagination, I could see your smile dawning, my little angel, your kind, friendly little smile; and my heart felt just the same as when I kissed you, Varyenka---do you remember that, my little angel? You know, my pet, I even thought that I could see you shaking your little finger at me? Were you, you naughty girl? I am certain that you will write about all of this in more detail in your letter.

Well, what do you think about our little trick with your curtain, Varyenka? Awfully nice, don't you think? Whether I am sitting down to work, lying down to sleep, or just waking up, I can always know that you are thinking about me, that I am on your mind, and that you yourself are happy and healthy. If you put down your curtain, that means, ``Goodbye, Makar Alekseyevich, time to sleep!'' If you raise it, that means, ``Good morning, Makar Alekseyevich, how did you sleep?''~or ``How is your health, Makar Alekseyevich? As for me, I am healthy and contented, thanks be to the Creator!'' You see, my sweet, how cleverly thought-out it is; no letters needed! Cunning, isn't it? And I thought up that little trick myself! What do you think of me coming up with these things, Varvara Alekseyevna?

I can report to you, my precious Varvara Alekseyevna, that everything was in good order with my sleep last night, contrary to my expectations, which is quite satisfying; even though I am in my new rooms, just after moving, and for some reason I can never sleep; something should always be just so and is not just so! I woke up this morning feeling like the bright falcon in the fairy tale---cheerful and happy! What a fine morning it is today, my precious! We opened our window; the sun is shining, the birds are chirping, the air is full of spring fragrances, and all of nature is coming to life---well, and everything else corresponded to that; everything was in order, spring-like. I even had a pleasant dream today, and all of my dreams were of you, Varyenka. I compared you with the birds of the air, created to delight humans and to adorn nature. Then I thought, Varyenka, that we humans, living in worry and agitation, should also be envious of the innocent and carefree happiness of the birds of the air---well, and so on, and so on, you know the rest; in other words, I was making all of these far-flung comparisons. I have a certain book, Varyenka, that has things like this in it, and everything is written there in detail. What I mean to write, my precious, is that there are all sorts of dreams. And now it is spring, and all these pleasant, witty, intricate thoughts and sentimental dreams come into my head; everything looks rosy. That is what I am trying to write; although all of this I got out of that book. There the author reveals his desire in verses and writes:

\begin{quotation}
	Would that I were a bird, a fearsome bird of prey!
\end{quotation}

And so on. There are other kinds of thoughts there, but forget about them. Where was it you were going this morning, Varvara Alekseyevna? I had not started out for the office yet and you had already fluttered out of your room, truly like a birdie of the air, and walked through the courtyard looking so happy. I was so happy just to look at you! Oh Varyenka, Varyenka! Don't you be sad; tears have never helped anyone's sorrow; that I know, my precious, that I know from experience. You have it quiet now, and your health has gotten a little better. And how is your Fyedora? What a kind woman she is! Will you write to me, Varyenka, and tell me how you and she are getting on and whether you are satisfied with everything? Fyedora is a bit grumpy; don't you pay any attention to that, Varyenka. Forget about that! She is so kind.

I have already written to you about Tereza over here---she is also a woman who is kind and true. I had been worrying so much about our letters! How could they make their way back and forth? And then the Lord sent Tereza to ensure our happiness! She is a kind, gentle, speechless woman. But our landlady is simply merciless. She puts her through the wringer like an old rag with all the work she gives her.

What a hole of a place I have fallen into, Varvara Alekseyevna! Well, it is an apartment! Before, I lived almost like a deaf person, as you know: meekly, quietly; if a fly was flying around my room, you could hear the fly. But here all we have is noise, shouting, melee! But I suppose you have no idea how everything is set up here. Imagine, for example, a long corridor, absolutely dark and not very clean. On your right there is a blank wall, and on the left is door after door, and these are the units, stretching in a line. Well, people rent these units, and they have one room each; there are two or three people living in some of them. Don't ask what kind of disorder this is---it is Noah's ark! However, the people seem nice, everyone is educated, learned. One of them is an official of some kind (he is somewhere in the literary field), a well-read person: he talks about Homer and Brambeus and all those other authors they have, he talks about everything---such a smart person! Two officers live here and are always playing cards. A warrant officer lives here; an English teacher lives here. Stop, let me amuse you, precious; I will describe them satirically in a future letter, in other words, who they are and how they live, with all the details. Our landlady is a very short old woman who is not very clean---she goes around all day in her slippers and dressing-gown and yells at Tereza all day. I live in the kitchen, or it would be much more accurate to say it this way: here next to the kitchen there is a room (and I should note that we have a clean, bright, very nice kitchen), a tiny little room, a humble little corner\ldots{} In other words, or how could I say it better, the kitchen is large and the outer wall has three windows, so I have a little partition from the wall perpendicular to that one, and in a way that makes another room, a supernumerary unit; it is spacious and comfortable, and I have a window---in a word, very comfortable. Well, that is my little corner. Now I don't want you to think, precious, that there is any other mysterious significance to all of this; you might say, oh, the kitchen! In other words, yes, I suppose I do live in that room behind a partition, but that is nothing; I live here all to myself, off to the side, I live here quietly without a word to anyone. I have set up a bed here for myself, a table, a bureau, a couple of chairs, and hung up an ikon. It is true, there are apartments that are better---there may be ones that are much better---but comfort is the important thing; I mean, this is all for my comfort, and don't you think it is for any other reason. Your window is across from me, on the other side of the courtyard; and the courtyard is narrow, I see you in passing sometimes---it all makes me happy, unlucky as I am, and it is cheaper. Over here, the worst room with its own table costs 35 rubles in notes. I can't afford it! But my apartment costs me seven rubles in notes, and the table is five rubles in silver:\footnote{Explain these conversions to the reader!} that comes out to 24 and a half, and I used to pay exactly 30, although I had to deny myself a lot of things; I couldn't always drink tea, but now I am economizing enough for both tea and sugar. You know, it is shameful somehow, my child, not to drink tea; the people here are so well-to-do, so it is shameful. You drink it for the benefit of others, Varyenka, for the sake of appearances, to hit the right note; I don't care myself, I am not particular. Do you suppose that leaves much for pocket money? After all, one does need something for boots, clothing, and the like. So there goes my whole salary. But I am satisfied, I am not grumbling. It is enough. It has been enough for a few years running; and sometimes there are special awards. Well, farewell, my little angel. I bought you a couple of potted plants, a little balsam and a geranium---very inexpensive. But perhaps you also like mignonettes? I can get mignonettes, just write and ask; you know, just write in as much detail as you can. However, don't you worry or think anything ill of me that I have rented an apartment like this. No, I was forced to do it for comfort's sake, and I was enticed into it by comfort alone. After all, precious, I am saving money, putting something aside; I already have a little put away. Don't think that I am so meek that a fly's wing could break me down. No, precious, I make no mistake about myself, and I have the kind of character that befits a person of resolute and peaceful spirit. Farewell, my little angel! I went on writing to you for almost two full sheets, and it has long been time for me to leave for work. I kiss your fingers, precious, and remain

\begin{flushright}
Your most humble servant and faithful friend,\\
	Makar Devushkin.
\end{flushright}

\begin{flushright}
	April \nth{8}
\end{flushright}

\noindent Dear Makar Alekseyevich!

Do you know, now I simply must quarrel with you! I swear, Makar Alekseyevich, you are kind, but it is so difficult for me to accept your gifts. I know what they cost you, the hardship and deprivation they cause you, having to forgo necessary things. How many times have I said to you that I do not need anything, anything at all; that I cannot even repay you for the blessings you have already showered on me. So why these potted plants? Well, the balsam plants are fine, but geraniums on top of that? All I have to do is say one careless word, for ex.~about that geranium, and then you go and buy it; it was expensive, yes? The flowers blooming on it are lovely! Little crosses of crimson petals. Where did you find such a pretty geranium? I put it in the middle of the window sill, in the most visible place; I am going to put a little bench on the floor there, and I'll put some more flowers on the bench; just let me make my own little fortune first! Fedora cannot get enough; it is just like heaven in our room---clean and bright! And why did you send the candy? In fact, I guessed from your letter that all was not well---everything about heaven, and spring, and perfume in the air, and little birds chirping. What is this, I thought, is this poetry or something? I mean, in fact, poetry is all that was missing from your letter, Makar Alekseyevich! Tender feelings and rose-colored dreams---all of that was there! I did not even think about the curtain; it probably hooked itself on something when I was putting the pot on the window sill; there you have it!

Oh, Makar Alekseyevich! Whatever you say, however you try to count up your income to mislead me, to show that it all goes to you alone, you cannot hide it from me or cover it up. It is clear that you are depriving yourself because of me. Where did you get the idea, for ex., of renting an apartment like that? You are constantly disturbed and harassed; it is cramped and uncomfortable. You love solitude, and that is the one thing you do not have there! You could live much better, judging by your salary. Fedora says that you used to live far better than you do now. Surely you have not lived your whole life alone, in hardship, without happiness or a kind word from a friend, renting corners from strangers? Oh, my dear friend, how I pity you! Look out for your health at least, Makar Alekseyevich! You say that your eyes are getting weak, well, then don't write by candle-light; why do you need to write at home? I am sure that your bosses know how zealous you are about your work without that.

I am begging you again, do not waste so much money on me. I know that you love me, but you are not rich\ldots{} I woke up happy today as well. I felt so good; Fedora has been working for a long time now, and she has found me work, too. It made me so happy; I went out just to buy some silk, then I got to work. All morning I felt light-hearted, I was so happy! But now all my thoughts are sad and black; my heart is heavy.

Oh, I know something is going to happen to me, my fate is sealed! It is difficult to be in such uncertainty, not to have my future set, not to be able to foresee what will happen with me. It is certainly frightening to look back. Everything there is so sorrowful that it tears my heart in two to remember it. I could spend a century crying over the evil people who ruined me!

It is dusk. Time to work. I would have liked to write to you about so many things, but I have no time, I need to finish my work. I must hurry. Of course, the letters are so nice; it makes things not quite so boring. But why don't you come to visit us? Why is that, Makar Alekseyevich? You are close, now, after all, and you can manage to find a spare moment now and then. Come visit, please! I saw your Tereza. She seems so sickly; I pitied her; I gave her 20 kopecks. Oh, yes! I nearly forgot: you should definitely write to me about how things are over there in as much detail as you can. What kind of people are around you there, and do you live agreeably with them? I want to know all of that. See that you write about that, definitely! Today I will turn up the corner on purpose. Go to sleep earlier; yesterday I saw your light on until after midnight. Well, farewell. Today was melancholy, and boring, and sad! What a day! Farewell.

\begin{flushright}
	Yours,\\
	Varvara Dobroselova
\end{flushright}

\begin{flushright}
	April \nth{8}
\end{flushright}

\noindent Dear Varvara Alekseyevna!

Yes, precious, yes, my child, it was just my rotten luck to have a day like this! Yes; you were poking fun at me, an old man, Varvara Alekseyevna! It is my fault, however, entirely my fault! I should not be starting off on amours and equivoques at my age, with only a wisp of hair left on my head\ldots{} I will say this, though, precious: people are funny sometimes, very funny. They will start talking about something and, by all that is holy, they get carried away sometimes! And what comes out of all this, what is the result? Nothing comes out of it at all, and the result is rubbish, Lord save me! I am not angry, precious, but I am annoyed when I remember everything I wrote to you so stupidly, in figurative language, just annoyed. And I left for work feeling as if I were wearing a brand-new overcoat; my heart was radiant. Out of nowhere, I felt like celebrating; I was happy! 